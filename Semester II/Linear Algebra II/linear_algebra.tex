\documentclass{article}
\usepackage{amsmath,amssymb,amsthm}
\newtheorem{theorem}{Theorem}
\title{Linear Algebra II}
\author{
\begin{tabular}{c}
Anurag (\texttt{bmat2508@isibang.ac.in})\\[0.6em]
Instructor: B.\,V.\ Rajarama Bhat
\end{tabular}
}
\date{}
\renewcommand{\thesubsection}{\arabic{subsection}}
\newcommand{\Col}{\operatorname{Col}}
\newcommand{\Null}{\operatorname{Null}}
\begin{document}
\maketitle
\section*{Class 2, January 7 2026 (Substitution)}  
We first show what is called the \textit{Fundamental Theorem of Linear Algebra}:\\
\begin{theorem}
If $M\in M_n(\mathbb{R})$, then $\mathbb{R}^n=C(A)\oplus N(A^{T})$.
\end{theorem}
\begin{proof}
Simple, just use the fact that for any $\mathbf{x}\in \mathbb{R}^n$, $\mathbf{x}^{T}\mathbf{x}=0 \implies \mathbf{x}=\mathbf{0}$
\end{proof}
Another result:
\begin{theorem}[Fredholm Alternative] 
    $\mathbf{a}\in C(A) \iff \mathbf{a}^{T}\mathbf{b}=0 \text{ }\forall \text{ }\mathbf{b} \in N(A^{T})$.
\end{theorem}
\begin{proof}
Simple, just rewrite $\mathbf{a}$ as in the previous theorem : $\mathbf{a}=A\mathbf{y}+\mathbf{x}$ 
\\for $\mathbf{y}\in \mathbb{R}^{n} 
\text{ and } \mathbf{x}\in N(A^{T})$. Then this follows easily.
The other direction is trivial.
\end{proof}
A good application of this is to check when a system of equations has a solution. 
The system $A\mathbf{x}=\mathbf{b}$ has a solution iff for any $\mathbf{x}\in N(A^{T})$ we have $\mathbf{b}^{T}\mathbf{x}=0$.  
\end{document}